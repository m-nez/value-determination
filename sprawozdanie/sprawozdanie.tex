\documentclass{article}
\usepackage{polski}
\usepackage[utf8]{inputenc}
\usepackage[pdftex]{graphicx}
\usepackage{indentfirst}
%\usepackage[margin=2.5cm]{geometry}
\usepackage{hyperref}
\usepackage{amsmath}
\DeclareGraphicsRule{*}{mps}{*}{}
\begin{document}
\title{Wyznaczanie wartości firm metodą porównawczą}
\author{Michał Nieznański, Stanisław Pawlak, Adrian Szewczyk}
\maketitle

\section{Wstęp}
Celem projektu miało być stworzenie progamu wyznaczającego wartość firm, w celu podejmowania
lepszych decyzji biznesowych.
\section{Program główny}
Program dokonuje analizy wsadowej jako zadanie Apache Spark.
W celu uruchomienia programu na konkretnym klastrze należy podać go jako argument programu.
Uruchomienie wygląda następująco:
\begin{verbatim}
$SPARK_HOME/bin/spark-submit --py-files analysis.py \
value_determination.py files/*.csv --master spark://hostname:port
\end{verbatim}
\$SPARK\_HOME to katalog, w którym znajduje się Spark.
Oprócz głównego skryptu należy załączyć wszystkie moduły, z których ten skrypt korzysta używając
opcji
\begin{verbatim}
--py-files
\end{verbatim}
programu spark-submit, w tym przypadku analysis.py.
Argumentem pozycyjnym programu value\_determination jest lista plików CSV zawierających dane z raportów
kwartalnych firm.
\subsection{Działanie}
Program wczytuje pliki do DataFrame, a następnie zrównolegla wykonanie analizy pomiędzy kwartały
poprzez sc.parallelize, a następnie map na otrzymanym RDD.
sc to SparkContext inicializowany na początku programu.
Wynik mapowania zwraca wyliczone ceny akcji firm w kwartałach.
\section{Wizualizacja}
Wyliczone ceny akcji są wizualizowne, a skuteczość analizy programu weryfikowana poprzez
przeprowadzenie symulacji giełdowej.
\section{Wnioski}
\begin{itemize}
	\item Wykorzystanie klastra Apache Spark do zadania o
		małych rozmiarach skutkuje narzutami czasowymi związanymi z komunikacją,
		o wiele rzędów wielkości przekraczającymi czas wykonania jako skrypt na jednej maszynie
	\item Apache Spark pozwala na łatwą skalowalność infrastruktury wykonującej zadania
\end{itemize}
\end{document}
