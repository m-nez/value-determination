\section{Wstęp}
Celem projektu miało być stworzenie progamu wyznaczającego wartość firm, w celu podejmowania
lepszych decyzji biznesowych.
\section{Uzyskiwanie danych}

Podstawowym zadaniem w big data jest uzyskiwaania danych do analizy, które najczęściej są otrzymywane przez crawler. Jest to program zbierający informacje o strukturze, stronach i treściach znajdujących się w internecie. Efekty pracy robota mogą być różne, w zależności od jego przeznaczenia, na przykład może on skanować wybrane witryny w celu zbudowania danych.

\subsection{Crawling w naszym projekcie}

W naszym przypadku robot chodził po stronie https://www.biznesradar.pl/ i pobierał z niej interesujące nas dane.
Wszystkie dane są zapisywane do pliku {INDEX}.csv w folderze raporty. Dane te otrzymujemy dla każdego kwartału, pomijając pięć pierwszych kwartałów od początku działalności.

\subsubsection{https://www.biznesradar.pl/raporty-finansowe-rachunek-zyskow-i-strat/ \{INDEX\},Q}
Z tej podstrony uzyskujemy następujące dane:
\begin{enumerate}
	\item Kwartał
	\item Przychody ze sprzedaży
	\item Techniczny koszt wytworzenia produkcji sprzedanej
	\item Koszty sprzedaży
	\item Koszty ogólnego zarządu
	\item Zysk ze sprzedaży
	\item Pozostałe przychody operacyjne
	\item Pozostałe koszty operacyjne
	\item Zysk operacyjny (EBIT)
	\item Przychody finansowe
	\item Koszty finansowe
	\item Pozostałe przychody (koszty)
	\item Zysk z działalności gospodarczej
	\item Wynik zdarzeń nadzwyczajnych
	\item Zysk przed opodatkowaniem
	\item Zysk (strata) netto z działalności zaniechanej
	\item Zysk netto
	\item Zysk netto akcjonariuszy jednostki dominującej
\end{enumerate}

\subsubsection{https://www.biznesradar.pl/wskazniki-wartosci-rynkowej/\{INDEX\},0}
\begin{enumerate}
	\item Kurs
	\item Ilość akcji
\end{enumerate}

\subsection{Etap filtracji danych}
Etap filtracji danych polegał na odrzuceniu pięciu pierwszych raportów kwartarnych dla każdej spółki.

\subsection{Wykorzystana technologia}
Crawler został napisany w języku C\# przy wykorzystaniu biblioteki HtmlAgilityPack.
\section{Program główny}
Program dokonuje analizy wsadowej jako zadanie Apache Spark.
W celu uruchomienia programu na konkretnym klastrze należy podać go jako argument programu.
Uruchomienie wygląda następująco:
\begin{verbatim}
$SPARK_HOME/bin/spark-submit --py-files analysis.py \
value_determination.py files/*.csv --master spark://hostname:port
\end{verbatim}
\$SPARK\_HOME to katalog, w którym znajduje się Spark.
Oprócz głównego skryptu należy załączyć wszystkie moduły, z których ten skrypt korzysta używając
opcji
\begin{verbatim}
--py-files
\end{verbatim}
programu spark-submit, w tym przypadku analysis.py.
Argumentem pozycyjnym programu value\_determination jest lista plików CSV zawierających dane z raportów
kwartalnych firm.
\subsection{Działanie}
Program wczytuje pliki do DataFrame, a następnie zrównolegla wykonanie analizy pomiędzy kwartały
poprzez sc.parallelize, a następnie map na otrzymanym RDD.
sc to SparkContext inicializowany na początku programu.
Wynik mapowania zwraca wyliczone ceny akcji firm w kwartałach.
\subsection{Analiza}
W trakcie analizy ze wszystkich dostępnych firm wybierane jest K najbardziej podobnych
firm ze zbioru danych.
W celu obliczenia podobieństwa wskaźniki firm są normalizowane.
Uzyskane w ten sposób wektory liczb z zakresu od 0 do 1 są porównywane pod względem
odległości euklidesowej do firmy docelowej, której kurs ma być obliczony.
\section{Wizualizacja}
Wyliczone ceny akcji są wizualizowne, a skuteczość analizy programu weryfikowana poprzez
przeprowadzenie symulacji giełdowej.

\subsection{Symulacja giełdowa}
Symulację giełdową robiliśmy na podstawie porównania zysku lub straty \textit{Buy And Hold} oraz zysku lub straty algorytmu sterowanego naszą wyceną firm. 
\section{Wnioski}
\begin{itemize}
	\item Wykorzystanie klastra Apache Spark do zadania o
	małych rozmiarach skutkuje narzutami czasowymi związanymi z komunikacją,
	o wiele rzędów wielkości przekraczającymi czas wykonania jako skrypt na jednej maszynie
	\item Apache Spark pozwala na łatwą skalowalność infrastruktury wykonującej zadania
\end{itemize}

\section{Wersjonowanie}
\begin{table}[ht!]
\centering
\begin{tabular}{|l|l|l|l|l|}
\hline
\textbf{Wersja}   & \textbf{Zmiany}        & \textbf{Kto}                                   & \textbf{Data} \\ \hline
0.1               & Utworzenie dokumentu   & \makecell{Stanisław Pawlak} & 14.06.2018    \\ \hline
0.2               & Opis programu głównego   & \makecell{Michał Nieznański} & 14.06.2018    \\ \hline
0.3               & Opis sposobu pozyskiwania danych   & \makecell{Adrian Szewczyk} & 14.06.2018    \\ \hline
\end{tabular}
\caption{Wersjonowanie}
\label{tab:ver}
\end{table}
